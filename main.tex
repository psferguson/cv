%% start of file `template.tex'.
%% Copyright 2006-2013 Xavier Danaux (xdanaux@gmail.com).
%
% This work may be distributed and/or modified under the
% conditions of the LaTeX Project Public License version 1.3c,
% available at http://www.latex-project.org/lppl/.


\documentclass[11pt,letterpaper, sans]{moderncv}        % possible options include font size ('10pt', '11pt' and '12pt'), paper size ('a4paper', 'letterpaper', 'a5paper', 'legalpaper', 'executivepaper' and 'landscape') and font family ('sans' and 'roman')

% modern themes
\moderncvstyle{banking}                            % style options are 'casual' (default), 'classic', 'oldstyle' and 'banking'

\moderncvcolor{purple}                                % color options 'blue' (default), 'orange', 'green', 'red', 'purple', 'grey' and 'black'
%\renewcommand{\familydefault}{\sfdefault}         % to set the default font; use '\sfdefault' for the default sans serif font, '\rmdefault' for the default roman one, or any tex font name
\nopagenumbers{}                                  % uncomment to suppress automatic page numbering for CVs longer than one page

\usepackage{xcolor}
% \definecolor{darkgreen}{RGB}{19, 84, 36}
\definecolor{darkgreen}{RGB}{5, 66, 21}
	
\usepackage[colorlinks = true,
            linkcolor = darkgreen,
            urlcolor  = darkgreen,
            citecolor = darkgreen,
            anchorcolor = darkgreen, unicode]{hyperref}
\usepackage{enumitem}
\let\oldhref\href
\renewcommand{\href}[2]{\oldhref{#1}{\bfseries#2}}

\usepackage{datetime}
\newdateformat{monthyeardate}{\monthname[\THEMONTH], \THEYEAR}

% character encoding
\usepackage[utf8]{inputenc}
\usepackage{fontawesome5}
\usepackage{tabularx}
\usepackage{ragged2e}
% if you are not using xelatex ou lualatex, replace by the encoding you are using
%\usepackage{CJKutf8}                              % if you need to use CJK to typeset your resume in Chinese, Japanese or Korean

% adjust the page margins
\usepackage[scale=0.8]{geometry}
\usepackage{multicol}
%\setlength{\hintscolumnwidth}{3cm}                % if you want to change the width of the column with the dates
%\setlength{\makecvtitlenamewidth}{10cm}           % for the 'classic' style, if you want to force the width allocated to your name and avoid line breaks. be careful though, the length is normally calculated to avoid any overlap with your personal info; use this at your own typographical risks...
\newcommand{\code}[1]{\texttt{#1}}
\usepackage{import}

% Replace bibentry with biblatex
\usepackage[backend=biber]{biblatex}
% Add the bibliography resource
\addbibresource{ferguson_publications.bib}

% Include aas_macros to define journal abbreviations
\usepackage{aas_macros}

% personal data
\name{Peter S.}{Ferguson}
\title{CV}                               % optional, remove / comment the line if not wanted
\address{Department of Astronomy}{University of Washington Seattle}{Box 351580, Seattle WA, 98195}% optional, remove / comment the line if not wanted; the "postcode city" and and "country" arguments can be omitted or provided empty
% \phone[mobile]{909-839-3097}                   % optional, remove / comment the line if not wanted
% \phone[fixed]{01234 123456}                    % optional, remove / comment the line if not wanted
%\phone[fax]{+3~(456)~789~012}                      % optional, remove / comment the line if not wanted
% \email{xpan1@swarthmore.edu}                               % optional, remove / comment the line if not wanted
% \homepage{shawnpan.me}                         % optional, remove / comment the line if not wanted
% \extrainfo{}                 % optional, remove / comment the line if not wanted
%\photo[64pt][0.4pt]{picture}                       % optional, remove / comment the line if not wanted; '64pt' is the height the picture must be resized to, 0.4pt is the thickness of the frame around it (put it to 0pt for no frame) and 'picture' is the name of the picture file
%\quote{Some quote}                                 % optional, remove / comment the line if not wanted

% to show numerical labels in the bibliography (default is to show no labels); only useful if you make citations in your resume
%\make@letter
%\renewcommand*{\bibliographyitemlabel}{\@biblabel{\arabic{enumiv}}}
%\makeatother
%\renewcommand*{\bibliographyitemlabel}{[\arabic{enumiv}]}% CONSIDER REPLACING THE ABOVE BY THIS

% bibliography with mutiple entries
%\usepackage{multibib}
%\newcites{book,misc}{{Books},{Others}}
  
\newcommand*{\customcventry}[7][.25em]{
  \begin{tabular}{@{}l} 
    {\bfseries #4}
  \end{tabular}
  \hfill% move it to the right
  \begin{tabular}{l@{}}
     {\bfseries #5}
  \end{tabular} \\
  \begin{tabular}{@{}l} 
    {\itshape #3}
  \end{tabular}
  \hfill% move it to the right
  \begin{tabular}{l@{}}
     {\itshape #2}
  \end{tabular}
  \ifx&#7&%
  \else{\\%
    \begin{minipage}{\maincolumnwidth}%
      \small#7%
    \end{minipage}}\fi%
  \par\addvspace{#1}}

\newcommand*{\customcvproject}[4][.25em]{
%   \vfill\noindent
  \begin{tabular}{@{}l} 
    {\bfseries #2}
  \end{tabular}
  \hfill% move it to the right
  \begin{tabular}{l@{}}
     {\itshape #3}
  \end{tabular}
  \ifx&#4&%
  \else{\\%
    \begin{minipage}{\maincolumnwidth}%
      \small#4%
    \end{minipage}}\fi%
  \par\addvspace{#1}}

\setlength{\tabcolsep}{12pt}

% Customize \fullcite appearance
\DeclareFieldFormat{title}{\textit{#1}}
\DeclareFieldFormat{author}{#1}
\DeclareFieldFormat{year}{#1}
\DeclareFieldFormat{journal}{\href{https://ui.adsabs.harvard.edu/abs/#1/abstract}{#1}}
\renewbibmacro*{author}{\printnames{author}}
\renewbibmacro*{title}{\printfield{title}}
\renewbibmacro*{year}{\printfield{year}}
\renewbibmacro*{journal}{\printfield{journal}}


%----------------------------------------------------------------------------------
%            content
%----------------------------------------------------------------------------------
\begin{document}
%\begin{CJK*}{UTF8}{gbsn}                          % to typeset your resume in Chinese using CJK
%-----       resume       ---------------------------------------------------------
\makecvtitle
\vspace*{-13mm}

\begin{center}
\begin{tabular}{ c c c}
 \faGlobe\enspace \href{https://peterferguson.space/}{\url{peterferguson.space}}& \faEnvelope\enspace \href{mailto://pferguso@uw.edu}{\url{pferguso@uw.edu}}& \faGithub\enspace \href{https://github.com/psferguson}{psferguson} \\  
\end{tabular}
\end{center}

\section{RESEARCH INTERESTS}
\vspace*{0mm}
\begin{center}
    \begin{tabular}{c c c c}
        Observational Cosmology & Milky Way (Sub)structure & Dark Matter & Tidal Streams \\
        Photometric Calibration & Large Scale Surveys & Instrumentation & Science Validation
    \end{tabular}
\end{center}


\section{APPOINTMENTS}

\begin{centering}
    \begin{tabular}{c l c}
        \vspace{1mm}
        Sep. 2024-Present & DiRAC Postdoctoral Fellow, DiRAC institute & U. of Washington
        \vspace{-1mm}\\
        %& UW Data Science Postdoctoral Fellow%, eScience Institute 
        && Seattle 
        \vspace{2mm}\\
        2021-2024 & Postdoctoral Research Associate  & U. of Wisconsin
        \vspace{-1mm}\\
        & Rubin Observatory commissioning/Observational cosmology &  Madison
        \vspace{2mm}\\
        2020-2021 & Universities Research Association (URA) Visiting Scholar & Fermilab
        \vspace{2mm}\\
        2016-2021 & Graduate Research Assistant: Astronomy \& Instrumentation &Texas A\&M\\
        %2016-2018 & Graduate Teaching Assistant: Astro-lab Coordinator/Instructor  & Texas A\&M\\
%\hfill Lab Coordinator/Lab Instructor for Astronomy Lab\hfill\,\\
        %2015 & Research Assistant with Kristin Chiboucas & Gemini Obs. North
    \end{tabular}
\end{centering}

\section{EDUCATION}

\textbf{Ph.D. in Astrophysics} \hfill 2021\\
Texas A\&M University \hfill \\
%\textbf{M.S. $\,$ in Astrophysics} \hfill 2019\\
Science: Jennifer Marshall \& Louis Strigari  \quad \quad \quad Instrumentation:  Darren DePoy %{\hfill $^\star$\footnotesize{Anticipated}}
\vspace*{2mm}

\textbf{B.S. in Astrophysics} \hfill 2013\\
Haverford College

\vspace*{-4mm}

\section{SCIENTIFIC COLLABORATIONS}
\begin{itemize}[itemsep=1pt, leftmargin=65pt]
    \item [2021-Present] \emph{Vera Rubin Observatory}: I am an in-kind contributor making significant contributions to many areas of the commissioning process. A few highlights:
    \begin{itemize}
        \item System Integration, Test, and Commissioning (SITCOM): I am one of the core members of the data analysis team helping provide rapid analysis of commissioning efforts (recently the telescope mount and main mirror cells). Additionally, I am helping to develop the tools we will use for the rest of commissioning. I have spent the last 6 months in Chile working closely with the construction team and acting as a volunteer observing specialist. 
        \item Science Verification and Validation (SVV): I have led ad-hoc analyses of Aux-Tel data, been a developer for \href{https://github.com/lsst/analysis_tools}{\code{analysis\_tools}} the metric framework for the Rubin Science Pipeline, helped to develop the science validation surveys for commissioning, and am the science lead for the commissioning science unit focused on object detection, quality flags, verification \& validation sample production, and survey property maps.
        \item Data Management (DM): I am one of the core members creating \href{https://github.com/lsst/analysis_tools}{\code{the\_monster}} an all sky reference catalog to bootstrap photometric calibrations for early operations. I also can process data and develop code for the rubin science pipelines. 
        
    \end{itemize}
    \item [2021-Present] \emph{Dark Energy Science Collaboration (LSST-DESC)} [\url{https://lsstdesc.org/}]\\
    \emph{Member}: Acting as a bridge between the Vera Rubin Observatory project and DESC to assist the Photometric Corrections, Science Release and Validation, and Commissioning working groups. \\
    \emph{Dark Matter Working Group Convener}: I am one of two conveners for this working group within DESC
    \item [2018-present] \emph{DECam Local Volume Exploration (DELVE) Survey} [\url{https://delve-survey.github.io}]\\
    \emph{Builder}: DELVE is a 3-year survey combining archival DECam data with 126 nights of dedicated observing. This survey looks to probe the small scale nature of dark matter by (1) searching for ultra faint MW satellites and stellar streams, (2) studying the satellite population and star formation history around the Large and Small Magellanic Clouds, and (3) deeply imaging around isolated Large Magellanic Cloud analogs to determine their satellite luminosity function. I have contributed to much of the calibration pipeline, data validation, and morphological classifier for our first data release DELVE-DR1 in early 2021. 
    
    \item [2016-present] \emph{Dark Energy Survey (DES)} [\url{https://www.darkenergysurvey.org}]\\
    %\emph{Member of Milky Way working group}: DES is a photometric survey over 5,000 deg in $griz$-bands to a depth of ~24th mag. I have contributed to validation of DES catalogs, and analyses looking for dwarf galaxies, stellar streams, and RR Lyrae in the DES datasets.  
    \item [2020-present] Southern Stellar Stream Spectroscopic Survey ($\mathcal{S}^5$) [\url{https://s5collab.github.io/}]\\
    %\emph{Member}: $\mathcal{S}^5$ is a spectroscopic survey run out of AAT to obtain spectra of stellar streams discovered in DES. I have contributed to target selection and determining stream membership probability of stars with 6D kinematic observations. 
\end{itemize}
\vspace*{-4mm}

\section{Publications}
\hspace{1.6in} Orcid: \href{https://orcid.org/0000-0001-6957-1627}{0000-0001-6957-1627} | ADS Library: \href{https://ui.adsabs.harvard.edu/user/libraries/rBhHJF5MTWqIJuR0n-PNKA}{Link}
%\hspace{1.6in} Orcid: \href{https://orcid.org/0000-0001-6957-1627}{0000-0001-6957-1627} | ADS Library: \href{https://ui.adsabs.harvard.edu/user/libraries/rBhHJF5MTWqIJuR0n-PNKA}{Link}
 
    % \begin{tabular}{c c}
    %      Orcid: \href{https://orcid.org/0000-0001-6957-1627}{0000-0001-6957-1627} & \href{https://ui.adsabs.harvard.edu/user/libraries/rBhHJF5MTWqIJuR0n-PNKA}{ADS List} 
    % \end{tabular}

\subsection{Primary Contributor}
\renewcommand{\labelitemi}{$$}
\begin{itemize}[itemsep=1pt]
    \item A. Drlica-Wagner, \textbf{P.S. Ferguson}, et al., \textit{The DECam Local Volume Exploration Survey Data Release 2}, 2022,  \href{https://ui.adsabs.harvard.edu/abs/2022ApJS..261...38D/abstract}{ApJS, 261, 2}
    \item \textbf{P.S. Ferguson}, N. Shipp, A. Drlica-Wagner, T.S. Li, et al., \textit{DELVE-ing into the Jet: a thin stellar stream on a retrograde orbit at 30 kpc}, 2022,  \href{https://ui.adsabs.harvard.edu/abs/2022AJ....163...18F/abstract}{AJ, 163, 1}
    \item K. Tavangar, \textbf{P.S. Ferguson}, N. Shipp, et al., \textit{From the Fire: A Deeper Look at the Phoenix Stream} 2022, \href{https://ui.adsabs.harvard.edu/abs/2021arXiv211003703T}{ApJ, 925,2}
    \item A. Drlica-Wagner, J. L. Carlin, D. L. Nidever, \textbf{P. S. Ferguson} et al., \textit{The DECam Local Volume Exploration Survey: Overview and First Data Release}, 2021, \href{https://ui.adsabs.harvard.edu/abs/2021ApJS..256....2D/abstract}{ApJS, 256, 1} 
    \item \textbf{P.S. Ferguson}, L.E. Strigari, \textit{Exploring the 3D structure of the Sagittarius dSph using RR-Lyrae}, 2020, \href{https://doi.org/10.1093/mnras/staa1404}{MNRAS, 495, 4}
    \item N. Shipp, A. Drlica-Wagner, E. Balbinot, \textbf{P. Ferguson}, et al., \textit{Stellar Streams Discovered in the Dark Energy Survey}, 2018, \href{https://doi.org/10.3847/1538-4357/aacdab}{ApJ, 862, 114}
\end{itemize}

\subsection{Co-Author}
\begin{itemize}[itemsep=1pt]
    \item C.Y. Tan, W. Cerny, A. Drlica-Wagner, ..., \textbf{P. S. Ferguson}, et al., \textit{A Pride of Satellites in the Constellation Leo? Discovery of the Leo VI Milky Way Satellite Ultra-faint Dwarf Galaxy with DELVE Early Data Release 3}, 2025, \href{https://ui.adsabs.harvard.edu/abs/2024arXiv240800865T}{ApJ, 979, 2}
    \item W. Cerny, A. Chiti, M. Geha, ..., \textbf{P. S. Ferguson}, et al., \textit{Discovery and Spectroscopic Confirmation of Aquarius III: A Low-Mass Milky Way Satellite Galaxy}, 2025, \href{https://ui.adsabs.harvard.edu/abs/2024arXiv241000981C}{ApJ, Accepted}
    \item A. Chiti, M. Mardini, G. Limberg, ..., \textbf{P. S. Ferguson}, et al., \textit{Enrichment by Extragalactic First Stars in the Large Magellanic Cloud}, 2024, \href{https://ui.adsabs.harvard.edu/abs/2024arXiv240111307C}{Nature Astr, accepted}
    \item S. Usman, A. Ji, T.S. Li, ..., \textbf{P. S. Ferguson}, et al., \textit{Multiple Populations and a CH Star Found in the 300S Globular Cluster Stellar Stream}, 2024, \href{https://ui.adsabs.harvard.edu/abs/2024MNRAS.tmp..241U}{MNRAS, accepted}
    \item M. McNanna, K. Bechtol, S. Mau, ..., \textbf{P. S. Ferguson}, et al., \textit{A search for faint resolved galaxies beyond the Milky Way in DES Year 6: A new faint, diffuse dwarf satellite of NGC 55}, 2023, \href{https://ui.adsabs.harvard.edu/abs/2023arXiv230904467M}{ApJ, 961,1}
    \item W. Cerny, A. Drlica-Wagner, T.S. Li, ..., \textbf{P. S. Ferguson}, et al., \textit{DELVE 6: An Ancient, Ultra-faint Star Cluster on the Outskirts of the Magellanic Clouds}, 2023, \href{https://ui.adsabs.harvard.edu/abs/2023ApJ...953L..21C}{ApJ, 953, 2}
    \item E. A. Zaborowski, A. Drlica-Wagner, F. Ashmead, ..., \textbf{P. S. Ferguson}, et al., \textit{Identification of Galaxy-Galaxy Strong Lens Candidates in the DECam Local Volume Exploration Survey Using Machine Learning}, 2023, \href{https://ui.adsabs.harvard.edu/abs/2023ApJ...954...68Z}{ApJ, 954, 1}
    \item W. Cerny, C.~E. {Mart{\'\i}nez-V{\'a}zquez}, A. Drlica-Wagner, ..., \textbf{P. S. Ferguson}, et al., \textit{Six More Ultra-Faint Milky Way Companions Discovered in the DECam Local Volume Exploration Survey}, 2022, \href{https://ui.adsabs.harvard.edu/abs/2022arXiv220912422C}{ApJ, 953, 1}
    \item Y. Gordon, C. O'Dea, S. Baum, ..., \textbf{P. S. Ferguson}, et al., \textit{Compact Steep Spectrum Radio Sources with Enhanced Star Formation Are Smaller Than 10 kpc}, 2023, \href{https://ui.adsabs.harvard.edu/abs/2023ApJ...948L...9G}{ApJL, 948, 1}
    \item W. Cerny, J.D. Simon, T.S. Li, ..., \textbf{P. S. Ferguson}, et al., \textit{Pegasus IV: Discovery and Spectroscopic Confirmation of an Ultra-Faint Dwarf Galaxy in the Constellation Pegasus}, 2022, \href{https://ui.adsabs.harvard.edu/abs/2022arXiv220311788C/abstract}{ApJ, 942, 2}
    \item T. S. Li, A. Ji, A. B. Pace, ..., \textbf{P. S. Ferguson}, et al., \textit{$S^5$: The Orbital and Chemical Properties of One Dozen Stellar Streams}, 2022,  \href{https://ui.adsabs.harvard.edu/abs/2021arXiv211006950L/abstract}{ApJ, 928, 1}
    \item C.~E. Mart{\'\i}nez-V{\'a}zquez, W. Cerny, A.K. Vivas, ..., \textbf{P. S. Ferguson}, et al., \textit{RR Lyrae Stars in the Newly Discovered Ultra-faint Dwarf Galaxy Centaurus I}, 2021, \href{https://ui.adsabs.harvard.edu/abs/2021AJ....162..253M/abstract} {AJ 162, 6}
    \item N. Shipp, D. Erkal, A. Drlica-Wagner, ..., \textbf{P. S. Ferguson}, et al., \textit{Measuring the Mass of the Large Magellanic Cloud with Stellar Streams Observed by S $^{5}$}, 2021 \href{https://ui.adsabs.harvard.edu/abs/2021ApJ...923..149S/abstract}{ApJ 923, 2}
    \item W. Cerny, A.~B. Pace,  and A. Drlica-Wagner, ..., \textbf{P. S. Ferguson}, et al., \textit{Eridanus IV: an Ultra-faint Dwarf Galaxy Candidate Discovered in the DECam Local Volume Exploration Survey}, 2021, \href{https://ui.adsabs.harvard.edu/abs/2021ApJ...920L..44C/abstract}{ApJ, 920, 2}
    \item K. M. Stringer, A. Drlica-Wagner, L. Macri, ..., \textbf{P. S. Ferguson}, et al., \textit{Identifying RR Lyrae Variable Stars in Six Years of the Dark Energy Survey},2021, Submitted to AAS Journals, \href{https://arxiv.org/abs/arXiv:2011.13930}{arXiv:2011.13930}
    \item W. Cerny, A. B. Pace, A. Drlica-Wagner, \textbf{P. S. Ferguson}, et al., \textit{Discovery of an Ultra-Faint Stellar System near the Magellanic Clouds with the DECam Local Volume Exploration (DELVE) Survey}, 2021, \href{https://doi.org/10.3847/1538-4357/abe1af}{ApJ, 910, 18}

    \item T. T. Hansen, A. H. Riley, L. E. Strigari, J. L. Marshall, \textbf{P. S. Ferguson}, J. Zepeda, and C. Sneden, \textit{A Chemo-dynamical Link between the Gjöll Stream and NGC 3201}, 2020, \href{https://doi.org/10.3847/1538-4357/ababa5}{ApJ, 901, 23}
    
    \item T. T. Hansen, J. L. Marshall, J. D. Simon, ..., \textbf{P. S. Ferguson}, et al., \textit{Chemical Analysis of the Ultra-Faint Dwarf Galaxy Grus~II. Signature of high-mass stellar nucleosynthesis}, 2020, \href{https://doi.org/10.3847/1538-4357/ab9643}{ApJ, 897, 183}
    
    \item S. Mau, W. Cerny, A. B. Pace, ..., \textbf{P. S. Ferguson}, et.al., \textit{Two Ultra-Faint Milky Way Stellar Systems Discovered in Early Data from the DECam Local Volume Exploration Survey}, 2020, \href{https://doi.org/10.3847/1538-4357/ab6c67}{ApJ, 890, 136}
    
    \item K. M. Stringer, J. P. Long, L. M. Macri, ..., \textbf{P. S. Ferguson}, et.al.,\textit{Identification of RR Lyrae stars in multiband, sparsely-sampled data from the Dark Energy Survey using template fitting and Random Forest classification}, 2019, \href{https://doi.org/10.3847/1538-3881/ab1f46}{AJ, 158, 16}

\end{itemize}
\subsection{SPIE}
\begin{itemize}[itemsep=1pt]

    \item B. Quint, F. Daruich ... \textbf{P. S. Ferguson}, et al., \textit{Rubin M1M3 support system dynamic performance}. 2024,  \href{https://doi.org/10.1117/12.3019268}{Proceedings of the SPIE, Volume 13094}
    \item G. Rodeghiero, L. Rosignoli, ... \textbf{P. S. Ferguson}, et al., \textit{The Vera C. Rubin's M2 support system integration and verification at the TMA}. 2024,  \href{https://doi.org/10.1117/12.3019210}{Proceedings of the SPIE, Volume 13094}
    \item F. Daruich, C. Aguilar ... \textbf{P. S. Ferguson}, et al., \textit{Rubin Observatory primary tertiary mirror cell assembly: final integration and commissioning}. 2024,  \href{https://doi.org/10.1117/12.3036881}{Proceedings of the SPIE, Volume 13094}
    \item B. Stalder, F. Munoz ... \textbf{P. S. Ferguson}, et al., \textit{Rubin Observatory Simonyi Survey Telescope integrated mount performance}. 2024,  \href{https://doi.org/10.1117/12.3019266}{Proceedings of the SPIE, Volume 13094}
    \item L. P. Guy, K. Bechtol, J. L. Carlin, E. Dennihy, \textbf{P. S. Ferguson}, et al., \textit{Faro: A framework for measuring the scientific performance of petascale Rubin Observatory data products}. 2022, \href{https://arxiv.org/abs/2206.15447}{Proceedings of the SPIE, Volume 12189}
    \item \textbf{P. S. Ferguson}, L. Barba, D. L. DePoy, L. M. Schmidt, J. L. Marshall, et al., \textit{Further development and testing of TCal: a mobile spectrophotometric calibration unit for astronomical imaging systems}. 2020,  \href{https://doi.org/10.1117/12.2562736}{Proceedings of the SPIE, Volume 11447}
    \item \textbf{P. S. Ferguson}, D. L. DePoy, L. Schmidt, J. L. Marshall, et al., \textit{Development of TCal: a mobile spectrophotometric calibration unit for astronomical imaging systems}, 2018, \href{https://doi.org/10.1117/12.2313752}{Proceedings of the SPIE, Volume 107023A}
\end{itemize}

% Generated Publications
\renewcommand{\labelitemi}{$$}
\subsection{Primary Contributor  (10)}
\begin{itemize}[itemsep=1pt]
    \item \underline{{Boone}, K.}, \textbf{Ferguson, P. S.}, {Tabbutt}, M., et al., \textit{{Robust Measurement of Stellar Streams Around the Milky Way: Correcting Spatially Variable Observational Selection Effects in Optical Imaging Surveys}}, \href{https://ui.adsabs.harvard.edu/abs/2025arXiv251007511B}{\textbf{arXiv/2510.07511}, 2025}
    \item \textbf{Ferguson, P. S.}, {Rykoff}, Eli S., {Carlin}, Jeffrey L., et al., \textit{The Monster: A reference catalog with synthetic ugrizy-band fluxes for the Vera C. Rubin observatory}, \href{https://doi.org/10.71929/rubin/2583688}{\textbf{DMTN}, 2025}
    \item {Tsiane}, Kabelo, {Mau}, Sidney, {Drlica-Wagner}, Alex, ..., \textbf{Ferguson, P. S.}, et al., \textit{{Predictions for the Detectability of Milky Way Satellite Galaxies and Outer-Halo Star Clusters with the Vera C. Rubin Observatory}}, \href{https://ui.adsabs.harvard.edu/abs/2025OJAp....8E..89T}{\textbf{The Open Journal of Astrophysics}, 2025}
    \item {Li}, Ting S., {Ji}, Alexander P., {Pace}, Andrew B., ..., \textbf{Ferguson, P. S.}, et al., \textit{{S $^{5}$: The Orbital and Chemical Properties of One Dozen Stellar Streams}}, \href{https://ui.adsabs.harvard.edu/abs/2022ApJ...928...30L}{\textbf{\apj}, 2022}
    \item \textbf{Ferguson, P. S.}, {Shipp}, N., {Drlica-Wagner}, A., et al., \textit{{DELVE-ing into the Jet: A Thin Stellar Stream on a Retrograde Orbit at 30 kpc}}, \href{https://ui.adsabs.harvard.edu/abs/2022AJ....163...18F}{\textbf{\aj}, 2022}
    \item {Drlica-Wagner}, A., \textbf{Ferguson, P. S.}, {Adam{\'o}w}, M., et al., \textit{{The DECam Local Volume Exploration Survey Data Release 2}}, \href{https://ui.adsabs.harvard.edu/abs/2022ApJS..261...38D}{\textbf{\apjs}, 2022}
    \item \underline{{Tavangar}, K.}, \textbf{Ferguson, P. S.}, {Shipp}, N., et al., \textit{{From the Fire: A Deeper Look at the Phoenix Stream}}, \href{https://ui.adsabs.harvard.edu/abs/2022ApJ...925..118T}{\textbf{\apj}, 2022}
    \item {Drlica-Wagner}, A., {Carlin}, J. L., {Nidever}, D. L., \textbf{Ferguson, P. S.}, et al., \textit{{The DECam Local Volume Exploration Survey: Overview and First Data Release}}, \href{https://ui.adsabs.harvard.edu/abs/2021ApJS..256....2D}{\textbf{\apjs}, 2021}
    \item {Shipp}, N., {Drlica-Wagner}, A., {Balbinot}, E., \textbf{Ferguson, P. S.}, et al., \textit{{Stellar Streams Discovered in the Dark Energy Survey}}, \href{https://ui.adsabs.harvard.edu/abs/2018ApJ...862..114S}{\textbf{\apj}, 2018}
\end{itemize}
\subsection{Co-Author  (25)}
\begin{itemize}[itemsep=1pt]
    \item {Choi}, Yumi, {Olsen}, Knut A. G., {Carlin}, Jeffrey L., ..., \textbf{Ferguson, P. S.}, et al., \textit{{47 Tuc in Rubin Data Preview 1. Exploring Early LSST Data and Science Potential}}, \href{https://ui.adsabs.harvard.edu/abs/2025ApJ...992...47C}{\textbf{\apj}, 2025}
    \item {Schechter}, Paul L., {Sluse}, Dominique, {Zaborowski}, Erik A., ..., \textbf{Ferguson, P. S.}, et al., \textit{{The DELVE Quadruple Quasar Search. I. A Lensed Low-luminosity Active Galactic Nucleus}}, \href{https://ui.adsabs.harvard.edu/abs/2025AJ....170..241S}{\textbf{\aj}, 2025}
    \item {Medoff}, Jonah, {Mutlu-Pakdil}, Bur{\c{c}}in, {Carlin}, Jeffrey L., ..., \textbf{Ferguson, P. S.}, et al., \textit{{DELVE-DEEP Survey: The Faint Satellite System of NGC 55}}, \href{https://ui.adsabs.harvard.edu/abs/2025ApJ...990..108M}{\textbf{\apj}, 2025}
    \item {Pace}, Andrew B., {Li}, T. S., {Ji}, A. P., ..., \textbf{Ferguson, P. S.}, et al., \textit{{Spectroscopic Analysis of Pictor II: a very low metallicity ultra-faint dwarf galaxy bound to the Large Magellanic Cloud}}, \href{https://ui.adsabs.harvard.edu/abs/2025OJAp....8E.112P}{\textbf{The Open Journal of Astrophysics}, 2025}
    \item {Tan}, C. Y., {Cerny}, W., {Drlica-Wagner}, A., ..., \textbf{Ferguson, P. S.}, et al., \textit{{A Pride of Satellites in the Constellation Leo? Discovery of the Leo VI Milky Way Satellite Ultra-faint Dwarf Galaxy with DELVE Early Data Release 3}}, \href{https://ui.adsabs.harvard.edu/abs/2025ApJ...979..176T}{\textbf{\apj}, 2025}
    \item {Cerny}, W., {Chiti}, A., {Geha}, M., ..., \textbf{Ferguson, P. S.}, et al., \textit{{Discovery and Spectroscopic Confirmation of Aquarius III: A Low-mass Milky Way Satellite Galaxy}}, \href{https://ui.adsabs.harvard.edu/abs/2025ApJ...979..164C}{\textbf{\apj}, 2025}
    \item {Martinez}, Michael N., {Gordon}, Yjan A., {Bechtol}, Keith, ..., \textbf{Ferguson, P. S.}, et al., \textit{{Finding Lensed Radio Sources with the Very Large Array Sky Survey}}, \href{https://ui.adsabs.harvard.edu/abs/2025ApJ...979..132M}{\textbf{\apj}, 2025}
    \item {Teixeira}, G., {Bom}, C. R., {Santana-Silva}, L., ..., \textbf{Ferguson, P. S.}, et al., \textit{{Photometric redshifts probability density estimation from recurrent neural networks in the DECam local volume exploration survey data release 2}}, \href{https://ui.adsabs.harvard.edu/abs/2024A&C....4900886T}{\textbf{Astronomy and Computing}, 2024}
    \item {Chiti}, Anirudh, {Mardini}, Mohammad, {Limberg}, Guilherme, ..., \textbf{Ferguson, P. S.}, et al., \textit{{Enrichment by extragalactic first stars in the Large Magellanic Cloud}}, \href{https://ui.adsabs.harvard.edu/abs/2024NatAs...8..637C}{\textbf{Nature Astronomy}, 2024}
    \item {Usman}, Sam A., {Ji}, Alexander P., {Li}, Ting S., ..., \textbf{Ferguson, P. S.}, et al., \textit{{Multiple populations and a CH star found in the 300S globular cluster stellar stream}}, \href{https://ui.adsabs.harvard.edu/abs/2024MNRAS.529.2413U}{\textbf{\mnras}, 2024}
    \item {McNanna}, M., {Bechtol}, K., {Mau}, S., ..., \textbf{Ferguson, P. S.}, et al., \textit{{A Search for Faint Resolved Galaxies Beyond the Milky Way in DES Year 6: A New Faint, Diffuse Dwarf Satellite of NGC 55}}, \href{https://ui.adsabs.harvard.edu/abs/2024ApJ...961..126M}{\textbf{\apj}, 2024}
    \item {Zaborowski}, E. A., {Drlica-Wagner}, A., {Ashmead}, F., ..., \textbf{Ferguson, P. S.}, et al., \textit{{Identification of Galaxy-Galaxy Strong Lens Candidates in the DECam Local Volume Exploration Survey Using Machine Learning}}, \href{https://ui.adsabs.harvard.edu/abs/2023ApJ...954...68Z}{\textbf{\apj}, 2023}
    \item {Cerny}, W., {Drlica-Wagner}, A., {Li}, T. S., ..., \textbf{Ferguson, P. S.}, et al., \textit{{DELVE 6: An Ancient, Ultra-faint Star Cluster on the Outskirts of the Magellanic Clouds}}, \href{https://ui.adsabs.harvard.edu/abs/2023ApJ...953L..21C}{\textbf{\apjl}, 2023}
    \item {Cerny}, W., {Mart{\'\i}nez-V{\'a}zquez}, C. E., {Drlica-Wagner}, A., ..., \textbf{Ferguson, P. S.}, et al., \textit{{Six More Ultra-faint Milky Way Companions Discovered in the DECam Local Volume Exploration Survey}}, \href{https://ui.adsabs.harvard.edu/abs/2023ApJ...953....1C}{\textbf{\apj}, 2023}
    \item {Gordon}, Yjan A., {O'Dea}, Christopher P., {Baum}, Stefi A., ..., \textbf{Ferguson, P. S.}, et al., \textit{{Compact Steep Spectrum Radio Sources with Enhanced Star Formation Are Smaller Than 10 kpc}}, \href{https://ui.adsabs.harvard.edu/abs/2023ApJ...948L...9G}{\textbf{\apjl}, 2023}
    \item {Cerny}, W., {Simon}, J. D., {Li}, T. S., ..., \textbf{Ferguson, P. S.}, et al., \textit{{Pegasus IV: Discovery and Spectroscopic Confirmation of an Ultra-faint Dwarf Galaxy in the Constellation Pegasus}}, \href{https://ui.adsabs.harvard.edu/abs/2023ApJ...942..111C}{\textbf{\apj}, 2023}
    \item {Mart{\'\i}nez-V{\'a}zquez}, C. E., {Cerny}, W., {Vivas}, A. K., ..., \textbf{Ferguson, P. S.}, et al., \textit{{RR Lyrae Stars in the Newly Discovered Ultra-faint Dwarf Galaxy Centaurus I}}, \href{https://ui.adsabs.harvard.edu/abs/2021AJ....162..253M}{\textbf{\aj}, 2021}
    \item {Shipp}, Nora, {Erkal}, Denis, {Drlica-Wagner}, Alex, ..., \textbf{Ferguson, P. S.}, et al., \textit{{Measuring the Mass of the Large Magellanic Cloud with Stellar Streams Observed by S $^{5}$}}, \href{https://ui.adsabs.harvard.edu/abs/2021ApJ...923..149S}{\textbf{\apj}, 2021}
    \item {Cerny}, W., {Pace}, A. B., {Drlica-Wagner}, A., ..., \textbf{Ferguson, P. S.}, et al., \textit{{Eridanus IV: an Ultra-faint Dwarf Galaxy Candidate Discovered in the DECam Local Volume Exploration Survey}}, \href{https://ui.adsabs.harvard.edu/abs/2021ApJ...920L..44C}{\textbf{\apjl}, 2021}
    \item {Stringer}, K. M., {Drlica-Wagner}, A., {Macri}, L., ..., \textbf{Ferguson, P. S.}, et al., \textit{{Identifying RR Lyrae Variable Stars in Six Years of the Dark Energy Survey}}, \href{https://ui.adsabs.harvard.edu/abs/2021ApJ...911..109S}{\textbf{\apj}, 2021}
    \item {Cerny}, W., {Pace}, A. B., {Drlica-Wagner}, A., \textbf{Ferguson, P. S.}, et al., \textit{{Discovery of an Ultra-faint Stellar System near the Magellanic Clouds with the DECam Local Volume Exploration Survey}}, \href{https://ui.adsabs.harvard.edu/abs/2021ApJ...910...18C}{\textbf{\apj}, 2021}
    \item {Hansen}, T. T., {Riley}, A. H., {Strigari}, L. E., ..., \textbf{Ferguson, P. S.}, et al., \textit{{A Chemo-dynamical Link between the Gj{\"o}ll Stream and NGC 3201}}, \href{https://ui.adsabs.harvard.edu/abs/2020ApJ...901...23H}{\textbf{\apj}, 2020}
    \item {Hansen}, T. T., {Marshall}, J. L., {Simon}, J. D., ..., \textbf{Ferguson, P. S.}, et al., \textit{{Chemical Analysis of the Ultrafaint Dwarf Galaxy Grus II. Signature of High-mass Stellar Nucleosynthesis}}, \href{https://ui.adsabs.harvard.edu/abs/2020ApJ...897..183H}{\textbf{\apj}, 2020}
    \item {Mau}, S., {Cerny}, W., {Pace}, A. B., ..., \textbf{Ferguson, P. S.}, et al., \textit{{Two Ultra-faint Milky Way Stellar Systems Discovered in Early Data from the DECam Local Volume Exploration Survey}}, \href{https://ui.adsabs.harvard.edu/abs/2020ApJ...890..136M}{\textbf{\apj}, 2020}
\end{itemize}
\subsection{SPIE  (7)}
\begin{itemize}[itemsep=1pt]
    \item \textbf{Ferguson, P. S.}, {Barba}, L., {DePoy}, D. L., et al., \textit{{Further development and testing of TCal: a mobile spectrophotometric calibration unit for astronomical imaging systems}}, \href{https://ui.adsabs.harvard.edu/abs/2020SPIE11447E..5UF}{\textbf{SPIE}, 2020}
    \item \textbf{Ferguson, P. S.}, {DePoy}, D. L., {Schmidt}, L., et al., \textit{{Development of TCal: a mobile spectrophotometric calibration unit for astronomical imaging systems}}, \href{https://ui.adsabs.harvard.edu/abs/2018SPIE10702E..3AF}{\textbf{SPIE}, 2018}
\end{itemize}
\subsection{Unrefereed  (2)}
\begin{itemize}[itemsep=1pt]
    \item {Carlin}, Jeffrey L., \textbf{Ferguson, P. S.}, {Vivas}, A. Katherina, et al., \textit{{An Outer-disk SX Phe Variable Star in Rubin Data Preview 1}}, \href{https://ui.adsabs.harvard.edu/abs/2025RNAAS...9..161C}{\textbf{Research Notes of the American Astronomical Society}, 2025}
    \item \textbf{Ferguson, P. S.}, {Shipp}, Nora, \textit{{The DECam Field of Streams: a deep view of the Milky Way halo}}, \href{https://ui.adsabs.harvard.edu/abs/2025arXiv250605469F}{\textbf{arXiv/2506.05469}, 2025}
\end{itemize}

\newpage
\section{Selected Talks/Posters}
\subsection{Invited}
\begin{itemize}[itemsep=1pt, leftmargin=28pt]
    \item [2025] ``\textit{Unveiling Dark Matter with Dwarfs and Stellar Streams in the Era of Roman \& Rubin}",\\
    Cosmic Cartography With Roman Space Telescope \hfill Talk
    \item [2025] ``\textit{Unveiling dark matter in the near-field from present (DES) and future (LSST) cosmological surveys}",\\
    AAS 245 DES-DESC splinter session \hfill Talk
    \item [2024] ``\textit{Seeking the nature of dark matter with the Milky Way halo and wide-field photometric surveys}", \\
    Astronomy Colloquium, University of Washington \hfill Talk
    \item [2024] ``\textit{LSST Overview}",\\
    Dwarf Galaxies, Stellar Clusters and Streams in the LSST era, Chicago, IL\hfill Talk
    \item [2024] ``\textit{Seeking the nature of dark matter with the Milky Way halo and wide-field photometric surveys}", \\
    Cosmology Seminar at Carnegie Mellon University, Pittsburgh, PA\hfill Talk
    \item [2024] ``\textit{Seeking the nature of dark matter with the Milky Way halo and wide-field photometric surveys}", \\
    Astronomy Seminar, Dartmouth, NH\hfill Talk
    \item [2024] ``\textit{Seeking the nature of dark matter with the Milky Way halo}", \\
    MiFA Colloquium, Minneapolis, MN\hfill Talk
    \item [2023] ``\textit{Seeking the nature of dark matter with the Milky Way halo}", \\
    Noirlab South Colloquium, La Serena Chile\hfill Talk
    \item [2023]``\textit{Plenary talk on dark matter working group}",
    Remote, DESC spring meeting\hfill Talk
    \item [2023] ``\textit{Seeking the nature of dark matter with the Milky Way halo}", \\
    UW Madison Astronomy Colloquium\hfill Talk
   %\item [2022] ``\textit{Adventures in Calibration-land}", DELVE Collaboration Meeting \hfill Talk
    \item [2022] ``\textit{Calibration of the DELVE survey}", DESC Photometric Corrections Working Group \hfill Talk
    % \item [2021] ``\textit{Using streams and surveys to constrain dark mater power spectrum}", \\\Observational Cosmology group UW Madison \hfill Talk
   % \item [2021] ``\textit{DELVE-ing into the Jet}", University of Chicago \emph{Funch} %presentation \hfill Talk 
   % \item [2021] ``\textit{The Messy Side of the Milky Way: using large surveys to explore our% Galaxy's halo}",\\
  % Texas A\&M University Joint Nuclear and Astrophysics Seminar \hfill Talk
  % \item [2020]  ``\textit{Stellar Streams in DELVE}",\\
  % DECam Local Volume Exploration (DELVE) Survey Collaboration Meeting \hfill Talk
  %\item [2020] ``\textit{Milky Way working group activities}",\\
  %Dark Energy Survey(DES) Collaboration Meeting \hfill Talk
\end{itemize}
\subsection{Contributed}
\begin{itemize} [itemsep=1pt, leftmargin=28pt] % Reduce space between items
\item [2025] ``\textit{Searching for semi-resolved dwarfs in the era of LSST with Synthetic Source Injection (SSI)}", \\Galactic Frontiers II \hfill Poster
\item [2023] ``\textit{Dark Matter with Rubin Observatory}", PCW \hfill Session organizer
\item [2022] ``\textit{Calibration of the DELVE survey}", DESC Collaboration Meeting \hfill Talk
\item[2021] ``\textit{The Jet stream in DELVE}" \\
Texas Section of the American Physical Society (TSAPS), virtual\hfill Talk

\item[2021] ``\textit{The Jet stream in DELVE}" \\
Streams21: Constraints on Dark Matter, Virtual \hfill Talk

\item[2020] ``\textit{Further development and testing of TCal: a mobile spectrophotometric calibration unit for\\ astronomical imaging systems}", 
SPIE Astronomical Telescopes + Instrumentation, Virtual \hfill Poster

\item[2019] ``\textit{Exploring the 3D structure of the Sagittarius dSph core using RR Lyrae}"\\
RRL/Cepheid, Cloudcroft, NM  \hfill Talk
 
\item[2019] ``\textit{Constraining the 3D structure of the Sagittarius dwarf galaxy using RR-Lyrae and simple\\ hierarchical
Bayesian modeling}" Workshop on Astronomy \& Statistics, Texas A\&M University \hfill Talk

\item [2018] ``\textit{RR-Lyrae in the Dark Energy Survey}",
Near-Field Cosmology with the Dark Energy Survey's\\
DR1 and Beyond, Kavali Institute for Cosmological Physics, University of Chicago \hfill Talk 

\item Texas A\&M Astronomy Symposium \hfill 25 Aug. 2018 \\
Talk: \textit{TCal: a mobile spectrophotometric calibration unit for astronomical imaging systems} \\
Texas A\&M University, College Station, TX

\item[2018] ``\textit{TCal: a mobile spectrophotometric calibration unit for astronomical imaging systems}"\\
SPIE Astronomical Telescopes + Instrumentation, Austin, Tx.\hfill Poster

\item [2017] ``\textit{K2F2: Two new medium K-band filters on FLAMINGOS-2 at Gemini South}"\\
Frank N. Bash Symposium, The University of Texas at Austin \hfill Poster
\end{itemize}
\section{PROPOSALS}
Below are the successful observing proposals I have been the PI for.
\begin{itemize}[itemsep=1pt, leftmargin=28pt]
    \item [2022] \emph{Probing the Milky Way using DECam and stellar streams}\\ DECam 2023A (2.5 nights of observations)
    \item [2022] \emph{Probing the Milky Way using DECam and stellar streams}\\ DECam 2022B (3.5 nights of observations)
    \item [2019] \emph{Probing the Dynamical Structure of Sagittarius}\\ VLT/FLAMES cycle 105 (0.5 nights of observations pushed to 2021 due to COVID)
    \item [2019] \emph{Probing the Dynamical Structure of Sagittarius}\\ Gemini south 2020A (18 hours of observations not taken due to COVID)
\end{itemize}
\section{OBSERVING EXPERIENCE}
\begin{itemize}[itemsep=1pt, leftmargin=2pt]
    \item \textbf{Rubin Observatory} Simonyi Survey Telescope \hfill 15 nights\\
    Assisting comissioning and integration efforts
    \item \textbf{Cerro Tololo Interr-American Observatory} Chile -- Blanco 4-meter telescope \hfill 14 nights\\
           Used DECam both in person and remotely
    \item \textbf{McDonald Observatory} TX, USA -- Harlan Smith 2.7-meter telescope \hfill 20 nights\\
           Used Tull coud\'e Echelle Spectrograph for R-Process Alliance Observing
    \item \textbf{Gemini South:} Chile -- 8-meter telescope \hfill 3 nights\\
           Commissioned 2 new filters on FLAMINGOS-2
    \item \textbf{Gemini North:} HI, USA -- 8-meter telescope \hfill 3 nights\\
           Operated Queue as part of work at Gemini
\end{itemize}

\section{AWARDS}
\begin{minipage}{\maincolumnwidth}%
	\small{
    	\begin{itemize}[itemsep=1pt, leftmargin=28pt]
          \item [2021] Spring TSAPS outstanding talk by a graduate student 
          \item [2020] Fall TSAPS outstanding talk by a graduate student 
          \item [2020] URA Visiting Scholar at Fermilab award (Sponsor: Alex Drlica-Wagner)
          \item []
		\end{itemize}}%
\end{minipage}%
\vspace{-4mm}      
\section{MENTORING}
\begin{itemize} [itemsep=1pt, leftmargin=65pt]
    \item [2022-2025] Kyle Boone, a physics major at UW Madison, has worked on using synthetic source injection (Balrog) and survey property maps to generate stellar weight maps for DES analyses. Kyle is now a grad student at Harvard. 
    \item [2023-2024] Miranda Gorsuch, a physics PhD student at UW Madison, has worked on data analysis for LSST commissioning.
    \item [2020-2021] Kiyan Tavanagar, an astrophysics major at University of Chicago, has worked on characterizing stellar streams found in DES. Kiyan is currently a graduate student at Columbia.   
    \item [2018-2021] Leo Barba, a physics major at Texas A\&M, has worked 3D printing and designing parts for TCal as well as helping to set up and run the instrument. Currently astronomy graduate student at UMass Amherst.  
    \item [2018] Sarah Hughes, an REU student at Texas A\&M, helped to design the LabView based software used to run TCal.  
\end{itemize}

% \section{References}
% \begin{itemize} [itemsep=1pt, leftmargin=65pt]
%     \item[-] Keith Bechtol 
% Alex Drlica-Wagner <kadrlica@fnal.gov> 
% Keith Bechtol <kbechtol@wisc.edu>
% Jennifer Marshall <marshall@tamu.edu> 
% \end{itemize}
\vfill
\begin{center}
    Last Updated: \monthyeardate\today
\end{center}
{\centering }
% Publications from a BibTeX file without multibib
%  for numerical labels: \renewcommand{\bibliographyitemlabel}{\@biblabel{\arabic{enumiv}}}% CONSIDER MERGING WITH PREAMBLE PART
%  to redefine the heading string ("Publications"): \renewcommand{\refname}{Articles}
% \nocite{*}
% \bibliographystyle{plain}
% \bibliography{publications}                        % 'publications' is the name of a BibTeX file

% Publications from a BibTeX file using the multibib package
%\section{Publications}
%\nocitebook{book1,book2}
%\bibliographystylebook{plain}
%\bibliographybook{publications}                   % 'publications' is the name of a BibTeX file
%\nocitemisc{misc1,misc2,misc3}
%\bibliographystylemisc{plain}
%\bibliographymisc{publications}                   % 'publications' is the name of a BibTeX file

%-----       letter       ---------------------------------------------------------

\end{document}


%% end of file `template.tex'.
